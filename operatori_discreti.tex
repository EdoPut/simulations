\section{Operatori discreti}

\begin{definition}{Identità}
	\begin{align}
		\identity : \C &\mapsto \C
		\\
		x &\mapsto x
	\end{align}
\end{definition}

\begin{definition}{Shift}
	\begin{align}
		\shift : \C &\mapsto \C
		\\
		x_n &\mapsto x_{n+1}
	\end{align}
\end{definition}

\begin{definition}{Differenza}
	\begin{align}
		\difference : \C &\mapsto \C
		\\
		x_n &\mapsto x_{n+1} - x_{n}
	\end{align}
	
	L'operatore \difference può essere rappresentato con $\shift - \identity$
\end{definition}

\begin{proposition}
	Gli operatori \identity, \shift, \difference commutano fra loro.
\end{proposition}

\begin{proposition}
	Le potenze dei 3 operatori sono facili da rappresentare
	
	\begin{equation}
		\identity^k = \identity
	\end{equation}
	
	\begin{equation}
		\shift^k = x_n \mapsto x_{n+k}
	\end{equation}
	
	\begin{equation}
		\difference^k
		=
		(\shift - \identity)^k
		=
		\sum
			\limits_{i=0}^{k}
			{ k \choose i }
			\shift^{k-i}
			(-\identity)^i
	\end{equation}
\end{proposition}

\begin{proposition}{Somme telescopiche}
	\begin{equation}
		\sum
			\limits_{n=n_0}^{N-1}
			\difference y_n
		=
		y_n  \bigg\rvert_{n_0}^{N}
	\end{equation}
\end{proposition}

\begin{proposition}
	Abbiamo una formula parente dell'integrazione per parti
	
	\begin{equation}
	\int_{x}^{y}
		(fg)'
		dx
	=
	f g
	\bigg\rvert_{x}^{u}
	=
	\int_{x}^{y}
		f'g
		dx
	+
	\int_{x}^{y}
		fg'
		dx
	\end{equation}
	
	Infatti vale
	
	\begin{equation}
		\sum
			\limits_{n=n_0}^{N-1}
			\difference (y_n z_n)
		=
		y_n z_n  \bigg\rvert_{n_0}^{N}
		=
		\sum
			\limits_{n=n_0}^{N-1}
			z_n \difference y_n
		\sum
			\limits_{n=n_0}^{N-1}
			y_{n+1} \difference z_n
	\end{equation}
\end{proposition}

\section{Problemi alle differenze}

Supponiamo di voler trovare $y_n$ funzione discreta incognita e sappiamo che data $g_n$ funzione discreta nota

\begin{equation}
	\difference y_n = g_n
\end{equation}

Dai risultati precedenti

\begin{equation}
	\sum
		\limits_{n=n_0}^{N-1}
		\difference y_n
	=
	\sum
		\limits_{n=n_0}^{N-1}
		g_n
	=
	y_n \bigg\rvert_{n_0}^{N}
\end{equation}

E quindi otteniamo che 

\begin{equation}
	y_N
	=
	y_{n_0}
	+
	\sum
		\limits_{n=n_0}^{N-1}
		g_n
\end{equation}

Anche senza la condizione iniziale $y_0$ possiamo comunque trovare delle soluzioni non universalmente determinate; se esiste l'operatore inverso della differenza $\difference^{-1}$ possiamo applicarlo all'equazione precedente.

\begin{equation}
	y_n
	=
	\difference^{-1}
	g_n
\end{equation}

Ma se $w_n$ è una funzione discreta costante (o a periodo 1) se la sommiamo a $y_n$ descritta in modo implicito

\begin{equation}
	y_n
	=
	\difference^{-1}
	(g_n)
	+
	w_n
\end{equation}

troviamo che $y_n$ soddisfa ancora l'equazione precedente

\begin{equation}
	\difference
	y_n
	=
	\difference(
		\difference^{-1}
		(g_n)
		+
		w_n)
	=
	g_n
	+
	\difference
		w_n
	=
	g_n
\end{equation}

\section{Equazioni alle differenze}

\begin{definition}
	L'equazione
	
	\begin{equation}
		y_{n+1} = y_n + g_n
	\end{equation}
	
	è una equazione alle differenze (lineare)
\end{definition}

Se conoscessimo l'operatore antidifferenza $\difference^{-1}$ potremmo calcolare la soluzione alle equazioni alle differenze lineari senza fare sommatorie

\begin{equation}
	\sum
		\limits_{n=n_0}^{N-1}
		g_n
	=
	\sum
		\limits_{n=n_0}^{N-1}
		\difference y_n
	=
	y_n \bigg \rvert_{n_0}^{N}
	=
	\difference^{-1} (g_n) \bigg \rvert_{n_0}^{N}	
\end{equation}

Di qualche funzione conosciamo la antidifferenza

\begin{example}
	\begin{align*}
		f(x) = x^n
		\\
		D f = n x^{n-1}
		\\
		D^{-1} f = \frac{x^{n+1}}{n+1} + c
		\\
		f^{-1} = \frac{1}{x^n} 
	\end{align*}
\end{example}

La controparte discreta è la pseudo-potenza o potenza discreta

\begin{definition}[Pseudopotenza]
	
	$\pseudopotenza{x}{n} = x(x-1) \dots (x-n+1)$
\end{definition}

Questa funzione si comporta come la potenza sotto l'operatore di differenza

\begin{align*}
	\difference \pseudopotenza{x}{n}
	=
	\pseudopotenza{x+1}{n} - \pseudopotenza{x}{n}
	\\
	=
		(x+1) &x (x-1) \dots (x-n+2) -
		\\
		&x (x-1) \dots (x-n+2)(x-n+1)
	\\
	=
		(\pseudopotenza{x}{n-1})(x+1 - (x-n+1))
	\\
	=
		\pseudopotenza{x}{n-1}(n)
\end{align*}

E sotto l'operatore antidifferenza

\begin{equation}
	\difference^{-1} \pseudopotenza{x}{n}
	=
	\frac{
		\pseudopotenza{x}{n+1}
	}{
		n+1
	}
	+
	w_n
\end{equation}

Inoltre ammette inversa

\begin{equation}
	\pseudopotenza{x}{-n}
	=
	\frac{
        1
	}{
		(x+n) (x+n-1) \dots (x+1)
	}
\end{equation}

C'è una relazione che lega le potenze alle pseudo-potenze

\begin{theorem}
	Per ogni $n \geq 1$ esitono numeri $S_{i}^{n}$ numeri di Stirling di seconda specie per cui vale
	
	\begin{equation}
		x^n
		=
		\sum
			\limits_{i=1}^{n}
				S_{i}^{n}
				\pseudopotenza{x}{i}
	\end{equation}
\end{theorem}

Quindi volendo calcolare le sommatorie come

\begin{equation}
	\sum
		\limits_{x=n_0}^{N-1}
		x^k
	=
	\difference^{-1} x^k \bigg \rvert_{x=n_0}^{N}
\end{equation}

Invece che calcolare le antidifferenze delle potenze calcoliamo le antidifferenze delle pseudopotenze, di cui abbiamo una formula esplicita.

\begin{example}
	\begin{align*}
		% partenza
		\sum
			\limits_{x=n_0}^{N-1}
			x^k
		&=
		% al posto delle antidifferenze di x^k
		% ci mettiamo le antidifferenze delle 
		% scritture di x^k tramite pseudopotenze
		\difference^{-1}(
			\sum\limits_{i=1}^{k}
				S_{i}^{k}
				\pseudopotenza{x}{i}
		)
		\bigg\rvert_{x=n_0}^{N}
		% portiamo l'operatore antidifferenza dentro
		\\
		&=
		\sum\limits_{i=1}^{k}
			S_{i}^{k}
			\difference^{-1}(
				\pseudopotenza{x}{i}
			)
		\bigg\rvert_{x=n_0}^{N}
		=
		% valutiamolo sulle pseudopotenze
		\sum\limits_{i=1}^{k}
			S_{i}^{k}
			\frac{
				\pseudopotenza{x}{i+1}
			}{
				i+1
			}
		\bigg\rvert_{x=n_0}^{N}
		\\
		% valutiamolo numericamente
		&=
		\sum\limits_{i=1}^{k}
			S_{i}^{k}
			\frac{
				\pseudopotenza{N}{i+1} - \pseudopotenza{n_0}{i+1}
			}{
				i+1
			}
	\end{align*}
\end{example}

\begin{theorem}
	Per ogni $n \geq 1$ la scrittura con i numeri di Stirling di seconda specie è caratterizzata da 
	
	$$ S_{1}^{n} = S_{n}^{n} = 1$$
	
	$$ S_{i}^{n+1} = S_{i}^{n} + i S_{i}^{n}$$
	
	Quindi è possibile costruire una tabella triangolare inferiore di termini positivi
		
\end{theorem}

